\documentclass[11pt,portuguese]{article}
\usepackage[utf8]{inputenc}
\usepackage[T1]{fontenc}
\usepackage{graphicx}
\usepackage{ctable}
\usepackage{longtable}
\usepackage{array}
\usepackage{textcomp}
\usepackage{marvosym}
\usepackage{wasysym}
\usepackage{latexsym}
\usepackage{amssymb}
\usepackage{float}
\usepackage{wrapfig}
\usepackage{soul}
\usepackage{amssymb}
\usepackage{hyperref}
\usepackage{pdfpages}
\usepackage{needspace}
\usepackage[rubberchapters,clearempty,pagestyles]{titlesec}
\usepackage[portuguese]{babel}
\usepackage[margin=.5cm,landscape]{geometry}
\usepackage{enumitem}
\usepackage{multicol}
\usepackage{textcomp}
\usepackage{titlesec}
\usepackage{filecontents}
\usepackage{natbib}
\usepackage{bibentry}
\usepackage{longtable}
\usepackage{tabularx}
\usepackage{ltxtable}
\usepackage{booktabs}
\usepackage{color, colortbl}
\usepackage{fancyheadings}
\usepackage{graphics}
\usepackage{setspace}
\usepackage{tabularx}

\newcommand{\specialcell}[2][c]{%
  \begin{tabular}[#1]{@{}l@{}}#2\end{tabular}}

\begin{document}
\begin{doublespace}
\noindent
{\bf Formulário de Avaliação de Linguagens (CATP 01)} - INF01121 - Modelos de Linguagem de Programação (Prof. Lucas Mello Schnorr)

\noindent
Aluno(s): \hrulefill \\
\noindent
\begin{tabularx}{\textwidth}{|cX|cX}
LP1: & \hrulefill & LP2: & \hrulefill \\
\end{tabularx}
\end{doublespace}

\noindent
\begin{tabularx}{\textwidth}{p{10.5cm}|c|c|X}
Característica & {\bf LP1} & {\bf LP2} & Observações \\\hline
\specialcell{{\bf Simplicidade}\\(poucos comandos, fáceis de entender)}   &&&\\\hline
\specialcell{{\bf Ortogonalidade}\\(poucas primitivas, todas as combinações \\entre primitivas são legais e significativas)}   &&&\\\hline
\specialcell{{\bf Estrutura de Controle}\\(variedade, adequabilidade)}   &&&\\\hline
\specialcell{{\bf Tipos de Dados}\\(variedade, expressividade, semântica, \\mecanismos de reaproveitamento)}   &&&\\\hline
\specialcell{{\bf Estruturas de Dados}\\(variedade, expressividade, semântica, \\mecanismos de reaproveitamento)}    &&&\\\hline
\specialcell{{\bf Suporte a Abstração de Dados}\\(variedade de mecanismos de \\definição, manipulação e reaproveitamento de dados)}     &&&\\\hline
\specialcell{{\bf Suporte a Abstração de Controle}\\(variedade de mecanismos de definição, \\manipulação e reaproveitamento de processos, subprogramas)}     &&&\\\hline
\specialcell{{\bf Expressividade}\\(poder e conveniência de operadores)}    &&&\\\hline
\specialcell{{\bf Checagem de Tipos}\\(analisar também tipagem estática ou dinâmica)}     &&&\\\hline
\specialcell{{\bf Restrições de \emph{Aliasing}}}   &&&\\\hline
\specialcell{Suporte ao {\bf Tratamento de Exceções}}      &&&\\\hline
\specialcell{{\bf Portabilidade}}     &&&\\\hline
{\bf Reusabilidade}    &&&\\\hline
{\bf Tamanho de Código}     &&&\\\hline
\end{tabularx}

\noindent
\begin{tabularx}{\textwidth}{X|X|X}
\specialcell{A. Avalie a legibilidade (mais informativa \\
e com carga semântica maior)? \\
1 = menos legível; 3 - mais legível. \\
\begin{tabularx}{\linewidth}{p{1cm}|X|X}
\specialcell{\{ \\ ... \\
\}}
&
\specialcell{
begin \\
... \\
end 
}
&
\specialcell{
procedure Hello is \\
begin \\
 ... \\
end Hello
}
\end{tabularx}
}

& 
\specialcell{B. Ainda, em relação à legibilidade, circule \\
a melhor opção? \\
\ \\
\begin{tabularx}{\linewidth}{X|X}
a = a / b; & int a = a / b;
\end{tabularx}
}


&

\specialcell{C. Em relação à expressividade, circule \\
a melhor opção? \\
\ \\
\begin{tabularx}{\linewidth}{X}
\specialcell{for(int i=0; i<length(v); i++)\\\ \ \ print(v[i]);} \\\hline
\specialcell{foreach i in v do \\\ \ \ print i;}
\end{tabularx}
}

\\
\end{tabularx}

\end{document}
